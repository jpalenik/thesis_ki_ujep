\section{UX - Uživatelská zkušenost}
UX (z anglického user experience) je klíčovým faktorem při vytváření interaktivních produktů a aplikací. Odráží to, jak se uživatel cítí při interakci s~produktem, systémem nebo službou. Zahrnuje vnímání užitečnosti, jednoduchosti použití a efektivity. Správné navržení UX je důležité pro většinu společností, designérů a tvůrců při vytváření a zdokonalování produktů.

Podle Nielsen Norman Group 'uživatelská zkušenost' zahrnuje všechny aspekty interakce koncového uživatele se společností, jejími službami a produkty. Mezinárodní standard ISO 9241 definuje UX jako "vnímání a reakce uživatele vzniklé z používání nebo očekávaného používání systému, produktu nebo služby". \cite{iso9241}

Raný vývoj UX lze vysledovat až do strojové éry 19. a počátku 20. století. Frederick Winslow Taylor a Henry Ford byli průkopníky ve výzkumu efektivity práce a nástrojů. Termín UX byl zaveden Donaldem Normanem v polovině 90. let. \cite{nngroup}
