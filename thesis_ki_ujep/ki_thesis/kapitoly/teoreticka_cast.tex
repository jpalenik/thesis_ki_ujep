\chapter{Základní teoretické koncepty}

Tato kapitola se zabývá analýzou základních teoretických konceptů, které jsou fundamentální pro pochopení motivace a obsahu této práce. Kapitola zahrnuje následující klíčové prvky:

\begin{itemize}
  \item \textbf{Strojové učení:}
  
  Zabývá se vysvětlením základů strojového učení a jeho různých typů na základě vstupních dat.

  \item \textbf{Klasifikace dat v obraze:}
  
  V této části jsou vysvětleny základy klasifikace dat v obraze, což je často klíčový úkol ve strojovém učení.

  \item \textbf{Workflow management:}
  
  Zaměřuje se na nastavení a analýzu pracovního postupu, který jsou zásadní pro efektivní zpracování dat a organizaci anotačních procesů.

  \item \textbf{UX - Uživatelská zkušenost:}
  
  Poskytuje úvod do UX návrhu, který je důležitý zhlediska použitelnosti aplikace.
    
  \item \textbf{Uživatelské role:}
  
  Vysvětluje funkci a důležitost uživatelských rolí v rámci systému..

  \item \textbf{Metodika vývoje:}
  
  Soustředí se na proces vývoje webové aplikace, která je jádrem této práce, od analýzy po implementaci a testování.

  \item \textbf{Použité technologie:}
  
  Představuje konkrétní technologie a nástroje, jež byly využity během vývoje aplikace.
\end{itemize}

\import{./teoreticka_cast/}{strojove_uceni.tex}
\import{./teoreticka_cast/}{klasifikace_dat_v_obraze.tex}
\import{./teoreticka_cast/}{workflow_management.tex}
\import{./teoreticka_cast/}{ux.tex}
\import{./teoreticka_cast/}{uzivatelske_role.tex}
\import{./teoreticka_cast/}{metodika.tex}
\import{./teoreticka_cast/}{pouzite_technologie.tex}
