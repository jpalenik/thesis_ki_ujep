\chapter{Úvod}

Tato bakalářská práce se zaměřuje na návrh a vývoj webové aplikace určené pro efektivní anotace a organizaci shromažděných dat, která jsou nezbytná pro strojové učení. Dalším důležitým aspektem je implementace různých uživatelských rolí, které umožní vícefázovou kontrolu nad označenými daty. 

Spolupráce s firmou Melown Technologies SE dala práci reálný kontext a umožnila aplikaci inovací v praxi. Motivací k vývoji byla nutnost optimalizovat proces shromažďování dat v porovnání s tradičními metodami, jako je ruční anotace v geografických informačních systémech (GIS). Projekt vycházel z potřeby efektivně zpracovávat velké objemy snímků, jejichž manuální anotace v GIS byla neefektivní a vyžadovala součinnost více expertů.

Tradiční metody anotací se ukázaly jako časově náročné a neposkytovaly adekvátní kontrolu nad prací anotátorů, což bylo zvláště problematické u projektů vyžadujících detailní shromažďování dat pro strojové učení. Proto bylo nezbytné vyvinout nové řešení, které by bylo flexibilní, škálovatelné, umožnilo efektivní management datových sad a nabídlo možnost sledování historie dat pro detekci regresí a provádění zpětných úprav.

Vývoj této aplikace je klíčový, zejména kvůli narůstající potřebě efektivně zpracovávat a kategorizovat rozsáhlá data pro strojové učení. Proces je výzvou nejen technicky, ale i organizačně. Proto je zásadní mít nástroj pro správu, kontrolu a rozdělování úkolů mezi uživatele s různými rolemi. Tato bakalářská práce přináší užitek nejen firmě Melown Technologies SE, ale i širší komunitě v oblasti strojového učení.

První část práce poskytuje teoretický základ o strojovém učení, klasifikaci dat a uvádí čtenáře do problematiky. Dále se věnuje workflow managementu, uživatelské zkušenosti (UX), metodikám vývoje softwaru a v závěru kapitoly jsou představeny zvolené technologie použité při vývoji aplikace.

Druhá část práce se zaměřuje na analýzu procesu anotace a podrobně rozebírá důležitost uživatelských rolí v systému. Kapitola také obsahuje důkladný popis správy dat a klíčových prvků aplikace, včetně představení editoru a datového modelu aplikace.
