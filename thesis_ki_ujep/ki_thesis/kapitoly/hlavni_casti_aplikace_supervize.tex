\subsection{Supervize}

Supervize představuje klíčový prvek aplikace, zejména v kontextu kontroly kvality anotací prováděných anotátory. Dále poskytuje celistvý přehled nad aktuálně prováděnou prací napříč všemi projekty. Skládá se z pěti hlavních částí, které jsou zaměřené na kontrolu aktuální práce, kontrolu práce jednotlivých anotátorů, přehled aktuálně odevzdaných obrázků ke kontrole a nakonec pro přehled všech akcí prováděných v souvislosti se správou systému. Rozdělení u funkce jednotlivých částí je následující:

\subsubsection{1. Poslední revize}

Tato část poskytuje supervizorovi nebo administrátorovi aktuální přehled o práci anotátorů. Jsou zde zobrazeny aktuálně rozpracované soubory, a to bez ohledu na konkrétní projekt, dataset nebo uživatele. Avšak, supervizor nebo administrátor má možnost použít filtry na uživatele, projekt a dataset, což umožňuje zaměřit se na případné problémy v postupu práce a analyzovat práci anotátorů ve specifických kontextech.

\subsubsection{2. Podle uživatele}
Zde je zpřístupněn rychlý a jednoduchý přístup k přehledu práce jednotlivých uživatelů. Supervizor může jednoduše vidět, na jakých souborech anotátor pracoval, jestli 

\subsubsection{3. Odevzdané}


\subsubsection{4. Schválené}
\subsubsection{5. Updaty (aktualizace) anotace}


Supervize se skládá z 5 hlavních částí, jejichž hlavní náplní je kontrola odevzdaných anotací. V případě uspokojivých výsledků je anotace schválena a označena jako schválená. V opačném případě se vrací anotátorovi k opravě. Zásadním prvkem tohoto procesu jsou rovněž poskytnutí zpětné vazby anotátorům s cílem minimalizovat chybovost a zvýšit přesnost anotací. Dále je nezbytné identifikovat potenciální chyby, které by mohly v budoucnu narušit kvalitu trénovacích dat pro neuronové sítě.

Další hlavní funkcí je možnost monitorovat efektivitu práce anotátorů. Vzhledem k jejich finančnímu ohodnocení je nutné aby svou práci odváděli. Efektivitu lze měřit například prostřednictvím analýzy počtu odevzdaných anotací, sledování dynamiky ukládání dat a vztahu mezi časovým vynaložením a počtem anotovaných oblastí.

