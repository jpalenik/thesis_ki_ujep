 %volby: 
% male × female
% czech × english (zatím funguje jen czech)
% a studijní program / obor 
% is_bc (nejvíc odladěno)
% api_bc
% api_ing
% edu_bc
% edu_ing

\documentclass[male,czech,{is_bc}]{kitheses}
\usepackage{ifthen}

\usepackage{amsmath,amssymb}
\usepackage{graphics}
\usepackage{color}
\usepackage{array}
\usepackage{longtable}
\usepackage{afterpage}
\usepackage{import}
\usepackage{minted}
\usepackage{listings}
\usepackage{microtype}  % přesnější typografie


% workaround for imcompatibility of czech babel and biblatex

\iftutex
\else
\usepackage{etoolbox}
\makeatletter
\newcommand\my@hsyphen{-}
\newcommand\my@apostroph{'}
\patchcmd\select@language{-}{\my@hyphen }{}{\fail}
\patchcmd\select@language{'}{\my@apostroph }{}{\fail}
\makeatother
\fi

\usepackage[style=iso-numeric,shortnumeration=true]{biblatex}
\addbibresource{thesis.bib}


% fonty lze měnit (detaily viz sekce fonty)
\iftutex
	\usepackage{fontspec}  % nastavení fontů pro LuaLaTeX a XeLaTeX
	\setmainfont{Libertinus Serif}
	\setsansfont{Libertinus Sans}
	\setmonofont[Scale=MatchLowercase]{Source Code Pro}
	\usepackage{unicode-math}
	\setmathfont{Libertinus Math}
\else
	\usepackage[utf8]{inputenc} % nastavení pro PDF LaTeX
	\usepackage[T1]{fontenc}
	\usepackage{libertinus}
	\renewcommand{\ttdefault}{pxtt}
\fi

\usepackage{csquotes} % uvozovky

% sazba ukázek kódu 

\usepackage{listings}

% ukázka pro nastavení balíku listings pro sazbu ukázek zdrojových kódů
\lstset{ %
  language=Python,                % the language of the code
  basicstyle=\small\ttfamily,    
  backgroundcolor=\color{white},   % choose the background color. You must add \usepackage{color}
  showspaces=false,                % show spaces adding particular underscores
  showstringspaces=true,           % underline spaces within strings
  showtabs=false,                  % show tabs within strings adding particular underscores
  frame=single,                    % adds a frame around the code
  tabsize=3,                       % sets default tabsize to 2 spaces
  breaklines=true,                 % sets automatic line breaking
  breakatwhitespace=false,         % sets if automatic breaks should only happen at whitespace
  keywordstyle=\bfseries,          % keyword style
  commentstyle=\rmfamily,       % comment style
  stringstyle=\itshape\color,   % string literal style
}

% barevné zvýraznění textů, které je nutno nahradit
\newcommand{\ZT}[1]{\colorbox{yellow}{\color{red}{#1}}}


% TOTO JE POTŘEBA ZMĚNIT !!!!!!
\newcommand{\nazevcz}{Aplikace pro anotaci fotografických snímků 
geografické povahy
}        % zde VYPLŇTE český název práce (přesně podle zadání!)
\newcommand{\nazeven}{Application for annotation of photographic images of a geographical nature}     % zde VYPLŇTE anglický název práce (přesně podle zadání!)
\newcommand{\autor}{Ján Páleník}           % zde VYPLŇTE své jméno a příjmení
\newcommand{\rok}{\the\year}                
\newcommand{\vedouci}{Ing. Jakub Trojánek}         
% zde VYPLŇTE jméno a příjmení vedoucího práce, včetně titulů
\newcommand{\vedouciDAT}{Ing. Jakubovi Trojánkovi}
% zde VYPLŇTE jméno a příjmení vedoucího práce, včetně titulů ve třetím pádě
                                                           

% zvětšuje o 23% vertikální okraje v tabulkách
\renewcommand{\arraystretch}{1.23}

% nastavení pro záhlaví (co nelze udělat v cls souboru)

\renewcommand{\chaptermark}[1]{\markboth{\arabic{chapter}. #1}{}}
\pagestyle{fancy}

% nastavení odkazů
\usepackage{url} % formátování URL, příkaz \url
\usepackage{varioref} % lepší interní odkazy na obrázky, apod. příkaz \vref
\usepackage[unicode=true,pdfusetitle,
 bookmarks=true,
 breaklinks=false,pdfborder={0 0 1},backref=false,colorlinks=false]{hyperref} % hypertextové odkazy v PDF
 
\newcommand{\UV}[1]{\quotedblbase#1\textquotedblleft}
 
% odstraňte pokud vám vadí absence zarovnání dole 
\raggedbottom

\newcommand{\obrazek}[4][0.8\textwidth]{%
    \begin{figure}[h!]
      \centering
      \resizebox{#1}{!}{\includegraphics{kapitoly/obrazky/#2}}
      \caption{#3}
      \label{#4}
    \end{figure}
}


%%%%%%%%%%%%%%%%%%%%%%%%%%%%%%%%%%%%%%%%%%%%%%
% vlastní začátek dokumentu
%%%%%%%%%%%%%%%%%%%%%%%%%%%%%%%%%%%%%%%%%%%%%%

\begin{document}
\thispagestyle{empty}
\begin{center}
{
\LARGE
\univerzita\\[16pt]
\fakulta
}

\vspace{2cm}
\resizebox{8.42cm}{!}{
\ifthenelse{\boolean{czech}}
 {\includegraphics{LOGO_PRF_CZ_RGB_standard.jpg}}
 {\includegraphics{LOGO_PRF_EN_RGB_standard.jpg}}
}

\vspace{2cm}
{
\Huge\sffamily
\nazevcz\par
\vspace{0.6cm}
\Large\scshape \ifthenelse{\boolean{bc}}{bakalářská}{diplomová} práce
}
\end{center} 
 
\vfill
{
\large
\begin{tabular}{>{\bfseries}rl}
    Vypracoval: 	& \autor\\
    Vedoucí práce: 	& \vedouci\\
&\\
Studijní program:       & \program\\
\ifthenelse{\boolean{api}}{Studijní obor:          & \obor\\}{}
\end{tabular} 
}
\vspace{1.5cm}
\begin{center}
  \Large\scshape   Ústí nad Labem \rok
\end{center}

\cleardoublepage
\thispagestyle{empty}
\pagecolor{yellow}
{\Large Namísto žlutých stránek vložte digitálně podepsané zadání kvalifikační práce poskytnuté vedoucím katedry.\\\
Zadání musí zaujímat právě dvě strany.
}

Zadání je nutno vložit jako PDF pomocí některého nástroje, který umožňuje editaci dokumentů (se zachováním elektronického podpisu).

V Linuxe lze například použít příkaz \texttt{pdftk}.

\clearpage
\thispagestyle{empty}
\afterpage{\nopagecolor}
~
\clearpage

\thispagestyle{empty} 
{\bfseries Prohlášení}

\vspace{0.5cm}
Prohlašuji, že jsem tuto \ifthenelse{\boolean{bc}}{bakalářskou}{diplomovou} práci vypracoval\ifthenelse{\boolean{feminum}}{a}{}
samostatně a použil\ifthenelse{\boolean{feminum}}{a}{}
jen pramenů, které cituji a uvádím v přiloženém seznamu literatury.

\vspace{0.5em}

Byl\ifthenelse{\boolean{feminum}}{a}{} jsem seznámen\ifthenelse{\boolean{feminum}}{a}{} 
s tím, že se na moji práci vztahují práva a povinnosti vyplývající ze zákona c. 121/2000 Sb., ve znění zákona c. 81/2005 Sb., autorský zákon, zejména se skutečností, že Univerzita Jana Evangelisty Purkyně v Ústí nad Labem má právo na uzavření licenční smlouvy o užití této práce jako školního díla podle § 60 odst. 1 autorského zákona, a s tím, že pokud dojde k užití této práce mnou nebo bude poskytnuta licence o užití jinému subjektu, je Univerzita Jana Evangelisty Purkyně v Ústí nad Labem oprávněna ode mne požadovat přiměřený příspěvek na úhradu nákladu, které na vytvoření díla vynaložila, a to podle okolností až do jejich skutečné výše.

\vspace{2em}

V Ústí nad Labem dne \today   \hfill Podpis: \makebox[4cm][s]{\dotfill}

\cleardoublepage
\thispagestyle{empty}
~
\vfill

\begin{flushright}
    Děkuji vedoucímu práce \vedouciDAT{}
     za neocenitelné rady\\a pomoc při tvorbě bakalářské práce.
\end{flushright}

\cleardoublepage

\textsc{\nazevcz}

\textbf{Abstrakt:}

Cílem této bakalářské práce je návrh webové aplikace pro anotování fotografických snímků opatřených geografickými údaji. Aplikace má podporovat workflow management a správu uživatelů a jejich rolí v rámci systému. Výsledné anotace budou sloužit jako vstupní data pro strojové učení ve formátu GeoJSON. Klíčovým aspektem aplikace proto je, aby zajistila, že tato vstupní data budou kvalitní ve smyslu minimální chybovosti při maximální kvantitě.

\textbf{Klíčová slova:} anotace, strojové učení, workflow management

\bigskip


\textsc{\nazeven}

\textbf{Abstract:}

This bachelor thesis aims to design a web application for annotating photographic images with geographic data. The application is to support workflow management and management of users and their roles within the system. The resulting annotations will serve as input data for machine learning in GeoJSON format. A key aspect of the application is therefore to ensure that this input data is of high quality in the sense of minimum error rate with maximum quantity.

\textbf{Keywords:} annotation, machine learning, workflow management

\tableofcontents

\import{./kapitoly}{uvod.tex}

\import{./kapitoly}{teoreticka_cast.tex}

\import{./kapitoly}{prakticka_cast.tex}

\import{./kapitoly}{zaver.tex}

\import{./kapitoly}{seznam_zkratek.tex}

\sloppy
\listoffigures % Seznam obrázků

\sloppy
\printbibliography[title=Seznam použitých zdrojů]

\appendix

\chapter{Externí přílohy\label{sec:ep}}

Externí přílohy této bakalářské práce jsou umístěny na adrese:\\ \url{https://github.com/jpalenik/thesis_ki_ujep}.

Na úložiští GitHub jsou uloženy tyto externí přílohy:

\begin{itemize}
\item \textbf{zdrojové kódy bakalářské práce}
\item \textbf{vybrané zdrojové kódy aplikace}
\item \textbf{zdrojové kódy UML diagramů}
\end{itemize}

Základní struktura úložiště:

\begin{table}[h]
\begin{tabular}{ll}
\textbf{ki-thesis.pdf} & text práce v PDF \\
\textbf{thesis\_ki\_ujep} & zdrojový kód práce v \LaTeX{}u \\
\textbf{diagrams} & zdrojové kódy UML diagramů \\
\textbf{source\_codes} & vybrané zdrojové kódy aplikace \\
\end{tabular}
\end{table}


\end{document}

