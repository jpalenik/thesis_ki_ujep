\chapter{Závěr}

Cílem této bakalářské práce bylo navrhnout a implementovat webovou aplikaci pro anotování obrázků leteckých a uličních snímků. Aplikace je nyní plně funkční a aktivně využívaná firmou Melown Technologies SE. Díky této aplikaci bylo označeno více než 1000 snímků, které potvrzují její významný přínos pro anotační procesy ve firmě.

Je třeba zdůraznit, že úspěšný vývoj a implementace aplikace byl významně ovlivněn agilním přístupem k vývoji softwaru a schopností flexibilně reagovat na měnící se potřeby a požadavky. Výsledná aplikace nejenže zvyšuje efektivitu anotačního procesu, ale také přispívá k lepšímu pochopení a správě obrazových dat, což je klíčové pro úspěšné využití v oblasti strojového učení.

Byl navržen workflow management, který je úzce propojen s uživatelskými rolemi v aplikaci. Tento přístup přispěl k výraznému zjednodušení procesů a umožnil lepší rozdělení odpovědnosti a víceúrovňovou kontrolu anotací. Díky aplikaci odpadla potřeba pracovníků se znalostí GIS systémů. Vyvinutý 2D editor je jednodušší na použití, a tedy umožňuje jeho využívání i uživateli bez znalostí GIS.

Časově náročnou částí práce byla analýza požadavků a návrh nastavení správného workflow managementu. Tuto práci jsem vytvářel jako jediný vývojář, co znamenalo, že jsem prováděl veškerou analýzu procesů a datového návrhu včetně implementace. Vycházel jsem ze svých předchozích zkušeností a následně návrh konzultoval v rámci firmy s mým nadřízeným, který mi poskytl cennou zpětnou vazbu. Tento proces nejenže podpořil hlubší porozumění firemních potřeb a procesů, ale také mi umožnil zlepšit komunikační dovednosti a schopnost prezentovat technické koncepty.

Jako hlavní jazyk pro tuto aplikaci byl zvolen Javascript z důvodu jeho jednotnosti s ostatními aplikacemi ve firemním prostředí. Tato volba byla podpořena i flexibilitou a rozšířeností Javascriptu, která zjednodušuje integraci a budoucí rozvoj aplikace. Možnost použití jazyka PHP jsem proto v práci nezmiňoval, neboť neodpovídala technologickému přístupu firmy a zaměření na vytvoření jednotné, integrované a dobře spravovatelné aplikace.

V současné době probíhá rozšiřování aplikace o modul pro anotování 3D dat ze skenerů LiDAR, což představuje další krok k evoluci a rozšiřování funkčnosti systému. Během vývoje došlo k řadě úprav a vylepšení, zejména v oblasti 2D editoru, což významně přispělo ke zlepšení pracovních postupů a zvýšení uživatelské přívětivosti.

Pro další vývoj aplikace se nabízí možnost automatizace procesů, především v oblasti přiřazování obrázků anotátorům. Tento krok by mohl dále zvýšit efektivitu a snížit manuální úsilí a náklady firmy potřebné ke správě a distribuci dat. Dále by bylo vhodné se zaměřit na rozšíření analytických schopností aplikace. To může zahrnovat pokročilé funkce pro sledování a hodnocení kvality anotací. Rovněž by bylo užitečné provést další výzkum v oblasti uživatelského rozhraní a zkušeností, aby byla zajištěna intuitivita a výkonnost aplikace při rostoucí komplexitě a rozsahu funkcí.
