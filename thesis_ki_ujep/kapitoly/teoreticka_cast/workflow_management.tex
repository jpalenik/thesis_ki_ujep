\section{Workflow management}

Workflow management je termín používaný pro definování procesu opakujících se činností. Zahrnuje popis procesu pro návrh, vytvoření, provedení a sledování pracovních postupů. Jeho úlohy zahrnují:
\begin{itemize}
    \item nastavení a úpravy kroků procesu,
    \item řešení problémů,
    \item přizpůsobení se měnícím podmínkám,
    \item vytváření dokumentačních požadavků,
    \item informování zainteresovaných stran,
    \item udržování standardizovaného komunikačního systému.
\end{itemize}

Důležitost workflow managementu spočívá v prevenci ztráty času a zdrojů, která může být způsobena neefektivními pracovními postupy, a má významný dopad na úspěšnost a efektivitu v jakémkoli odvětví. Správně nastavený systém workflow managementu vede k minimalizaci a efektivnějšímu průběhu procesů. \cite{van_der_aalst_workflow_2004} 

Příkladem workflow může být plánování v rámci později zmíněné agilní metodiky vývoje -- scrum viz obrázek \vref{fig:scrum}.

\obrazek[\textwidth]{scrum.png}{scrum}{fig:scrum}

\subsection{Metody workflow managementu}

\subsubsection{Vizuální reprezentace a modelování}
Workflow management často zahrnuje použití vizuálních reprezentací. Například diagramy toku práce, které ukazují sekvenční a paralelní procesy. Tyto modely mohou zahrnovat šipky a další grafické prvky pro znázornění toku informací a pracovních úkolů. Vizualizace pracovních procesů umožňuje lepší porozumění a identifikaci potenciálních úzkých míst a zlepšení procesů.

\subsubsection{Rozlišování rolí a odpovědností}
V~rámci workflow managementu jsou rozlišovány různé role. V kontextu této práce jsou to role účetní, anotátora, supervizora a administrátora, každá s~vlastními odpovědnostmi a úkoly. Správná koordinace těchto rolí je nezbytná pro efektivní tok práce a zajištění, že každý úkol je správně adresován a vykonáván.

\subsubsection{Integrace technologie a informačních systémů}
Moderní workflow management často zahrnuje integraci technologických nástrojů a informačních systémů, jako jsou systémy pro správu dokumentů, databáze a softwarové aplikace. Tyto nástroje a systémy usnadňují automatizaci, sledování a analýzu pracovních procesů, což vede ke zvýšení efektivity a snížení chyb v rámci organizace. \cite{muehlen_workflow-based_2004}
