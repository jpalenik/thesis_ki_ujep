\section{Metodika vývoje}

Pro tento projekt byl zvolen agilní přístup vývoje. Agilní metodika vývoje představuje moderní přístup k softwarovému inženýrství, který klade důraz na flexibilitu, spolupráci a schopnost přizpůsobit se měnícím se podmínkám. Agilní vývoj je založen na principech obsažených v Agilním Manifestu pro Softwarový Vývoj a zdůrazňuje následující body: \cite{agilemanifesto} \cite{mijacobs_planovani_2023}

\begin{itemize}
  \item \textbf{Rychlé a iterativní vývojové cykly:} 
  
  Projekt je organizován do krátkých časových úseků nazývaných "sprinty," během nichž vývojový tým pracuje na konkrétních úkolech a funkcích. Tento přístup umožňuje flexibilní adaptaci k novým požadavkům a změnám.

  \item \textbf{Minimal viable product (MVP):} 
  
  Neboli minimalní životaschopný produkt je koncept vývoje produktu, který se zaměřuje na vytvoření produktu s~dostatečnými funkcemi, které splňují základní potřeby uživatelů a umožňují shromáždění zpětné vazby pro další iterace a vylepšení. Příkladem může být zadání výroby automobilu. Místo toho, aby byl zákazník nespokojený do doby, než se automobil celý vyrobí, tak se mu nejdříve dodá koloběžka. Poté pokračuje vývoj, v~další iteraci dostane motorku a nakonec auto. Stejný postup se dá aplikovat při vývoji aplikací.

  \item \textbf{Spolupráce a komunikace:} 
  
  Klade silný důraz na týmovou spolupráci a komunikaci mezi všemi zainteresovanými stranami, včetně zákazníků a uživatelů. Pravidelná komunikace a spolupráce přispívají k lepšímu porozumění požadavkům a zajišťují, že výsledný produkt nejlépe odpovídá potřebám uživatelů.

  \item \textbf{Flexibilita:} 
  
  Agilní metodologie vývoje poskytuje prostor pro adaptaci směru projektu v reakci na nově získané informace. Tato schopnost je obzvláště cenná v případech, kdy jsou požadavky nejasné nebo se mění.

  \item \textbf{Schopnost reagovat na změny:} 
  
  Agilní vývoj poskytuje rámec pro rychlé reagování na změny, který je důležitý pro projekty, kde se požadavky mohou vyvíjet nebo rozšiřovat s časem.

\end{itemize}
